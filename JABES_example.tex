%% This is an example file to show how to use JASA_manu.sty and related 
%% files written by Eric C. Anderson.  It's sort of a fly-by-night endeavor,
%% but I got it to work for me!!

\documentclass[11pt]{article}

\usepackage{graphicx}
\usepackage{endfloat}
\usepackage{amssymb}


%% THE NEXT TWO LINES INSERT THE PACKAGES FOR JASA FORMAT:
\usepackage[default]{jasa_harvard}    % 	for formatting citations in text
\usepackage{JABES_manu}


%% CHANGING THE 'AND' IN THE HARVARD BIBLIOGRAPHY PACKAGE TO WHAT IT OUGHT TO BE
\renewcommand{\harvardand}{and}


%% A FEW NEWCOMMANDS FOR THE CONTENT IN THE EXAMPLE
%% NOT CRUCIAL TO THE USE OF JASA_manu.sty
\newcommand{\Dir}{\mathrm{Dir}}
\newcommand{\ceil}[1]{\lceil #1 \rceil}
\newcommand{\thh}{^\mathrm{th}}
\newcommand{\modtwo}{\mathrm{[mod~2]}}
\newcommand{\thetaof}[2]{\theta \langle #1;#2\rangle}
\newcommand{\Mpa}{M_\mathrm{P,A}}
\newcommand{\Ma}{M_\mathrm{A}}
\newcommand{\rjaccept}{\mathcal{A}}



%% Here is a hand-formatted bibliography reference list entry.
%% The command should be included in the NOTE field of the entry in the .bib file
\newcommand{\RafteryDiscByHand}{Discussion of ``Model determination using predictive distributions
			   with implementation via sampling-based methods,'' by A.~E~Gelfand, D.~K.~Dey, and H.~Chang, in {\em Bayesian
Statistics
  4}, eds. J.~M. Bernardo, J.~O. Berger, A.~P. Dawid, \harvardand\ A.~F.~M.
  Smith, Oxford: Oxford University Press, pp.~147--167. }

\begin{document}



\title{Detecting Differential Expressed Genes in RNA-Seq Data \\ 
Accounting for Nuisance Covariates}
\author{Dan Nettleton  and Yet Nguyen\\
Department of Statistics \\ 
Iowa State University, Ames, IA 50010\\ 
email: \texttt{ntyet@iastate.edu} }

\maketitle


%% ABSTRACT

\newpage
\begin{center}
\textbf{Abstract}
\end{center}

This file, ``JASA\_example.tex," is just excerpted from a manuscript I wrote in the summer of 2000.
The content will appear totally discontinuous, but the use of various sectioning commands and citation commands should be
evident from it, and any other user should be able to use it as a template .tex file for using 
JASA\_manu.sty.  
Biologists regularly encounter populations of organisms with disparate ancestries.  Untangling the
composition of such populations is a problem for conservation biologists and wildlife
managers.  In many cases the population under question is known to consist of individuals from two
different subpopulations and their hybrids.  This occurs, for example, in hybrid zones between two species or
in regions recently colonized by exotics capable of reproducing with resident inhabitants.  This paper develops
techniques using multilocus genetic data for Bayesian clustering of individuals to purebred or
genetically-mixed categories.  The method relies on a novel application of the forward-backward recursions
in a two-component, finite mixture model.  Though developed in the context of the genetic
admixture problem, these calculations are relevant more generally to Bayesian inference in finite mixtures; they may
potentially improve mixing of the Gibbs sampler in such contexts.  
The technique is applied to genetic data on the Scottish wildcat, {\em Felis~sylvestris}, a protected species
whose distinctness from domestic housecats has been questioned.  A high proportion ($\approx .60$) of the
wild-living cats from which the sample was drawn are arguably purebred {\em F.~sylvestris}.  

Using the Bayes factor, we compare our new model, which allows for both purebred and admixed individuals, to
a model in which all individuals are assumed genetically admixed to some degree.  It is difficult to accurately compute
the marginal likelihood directly in these models, so we compute the Bayes factor by reversible-jump MCMC\@. The approach
follows from the original MCMC formulation of the problem, and should help to illustrate ways in which
reversible-jump methods may be implemented for comparisons between a small set of closely-related models.   

\vspace*{.3in}

\noindent\textsc{Keywords}: {Forward-backward recursion, Gibbs sampler, reversible jump, MCMC, hybrid zone}

\newpage

\section{Introduction}
% 
% 
% \citeasnoun{Pritchardetal2000} propose a versatile model for genetic
% inheritance in admixed populations and use it in 
% Bayesian analyses of population structure in several different
% species.   
% Before I go any further, let's look at a ``do-it-by-hand" citation/bibliography
% entry that was defined above: \cite{Raftery1992}.  A limitation of this model, however, is that it
% assumes every individual is admixed to some degree.  In many
% situations, such as with populations spanning hybrid zones, there is
% reason to expect both purebred and admixed individuals.  A probability
% model to accommodate such scenarios will include elements both of
% genetic mixture models and genetic admixture models.  
% In this paper I extend the methods of \citeasnoun{Pritchardetal2000} to handle
% explicitly  purebred individuals.  In section~\ref{sec:Adm}, I review mixture and admixture formulations for modeling
% population structure.  
% 
% In Section~3, I develop a method for making
% joint, Gibbs updates of large blocks of variables in \possessivecite{Pritchardetal2000} model. 
% The method uses the fact that the latent allocation variables of an i.i.d.~finite mixture, with a Dirichlet
% prior on mixing proportions can be shown to follow a hidden Markov chain, after integrating out the mixing 
% proportions.
% This computation facilitates MCMC simulation in a model, described later, that
% allows for both purebred and admixed individuals.  
% Additionally, I describe in the Discussion how such a method could help
% the Gibbs sampler to escape from trapping states \cite{Robert1996}
% encountered in finite mixture problems. 
% 
% I apply these techniques to data on the Scottish wildcat {\em Felis
% sylvestris}. In Scotland, {\em F.~sylvestris} evolved for thousands of
% years with little or no genetic exchange with cats in continental
% Europe.  Within the last 2,000 years these Scottish cats have suffered
% population declines due to human influences and have been exposed to
% possible interbreeding with domestic cats.  It can be difficult to
% distinguish {\em F.~sylvestris} from domestic cats on the basis of
% morphological characters alone and conservation biologists are concerned
% that the wild-living cats in Scotland may now represent an admixture of
% {\em F.~sylvestris} and domestic cats.  The data were previously analyzed
% by \citeasnoun**{Beaumontetal} using the method of \citeasnoun{Pritchardetal2000}.  
% However, this analysis does not address the issue of particular
% interest---that of 
% estimating the proportion of purebred
% {\em F.~sylvestris} individuals in the population.  Nor does that analysis allow estimation
% of posterior probabilities that particular individuals in the sample are purebred cats.  These questions about
% the Scottish wildcat population are similar to those for many species of conservation interest to which the
% present methods apply.  
% 
% Finally, using reversible-jump MCMC, it is possible to compute the Bayes factor for
% comparing the new, expanded model to that of \citeasnoun{Pritchardetal2000} given the Scottish
% cat data.  While the reversible-jump sampler allows  estimation of the true Bayes factor, 
% it is also possible to compute the ``pseudo-Bayes factor" \cite{Gelfandetal1992}, and assess how
% accurately that estimates the Bayes factor.
% 
% 
% 
% 
% \section{A Model with Admixed Individuals}
% \label{sec:Adm}
% With $\theta$ and $y$ defined as in the previous section, the model of
% \citeasnoun{Pritchardetal2000} is quickly described.   Now, $j$ indexes the $J$
% conceptual ``gene pools" or ``historical subpopulations" from which
% individuals may be descended.  Allowing for admixed individuals requires
% a different model of genetic inheritance, which, in turn, requires
% different latent variables.  The $i\thh$ individual in the
% sample gets  a vector of probabilities $q_i = (q_{i1},\ldots,q_{iJ})$, 
% $\sum_{j=1}^J q_{ij} = 1$, which are the unobserved proportions of that
% individual's genome descended from each of the $J$ gene pools. 
% Also, let
% $w_i = (w_{i1},\ldots,w_{i2L})$ be a vector of unobserved allocation
% variables which is parallel to the the vector of allelic types
% $y_i$.  Hence, $w_{it}=j$  indicates that the $(t~\modtwo +1)\thh$
% allele at the $\ceil{t/2}\thh$ locus in the $i\thh$ individual is from
% the $j\thh$ gene pool.   Given $w_{it} = j$ the type of allele is assumed
% to be drawn randomly according to $\theta_j$.  Under this model
% \begin{eqnarray}
% p(y_i|\theta,w_i) = \prod_{t=1}^{2L} \thetaof{w_{it}}{y_{it}}
% \end{eqnarray}
% independently for each $i$.
% By assigning the prior $q_i\sim\Dir(\alpha,\ldots,\alpha)$,  
% $i=1,\ldots,N$, and the hyperprior $\alpha \sim \mathrm{Uniform}(0,A]$,
% \citeasnoun{Pritchardetal2000}'s model is obtained.  In effect this is a 
% hierarchical model for $N$ different finite mixtures---the genes carried
% by the $i\thh$ individual are a sample from a mixture with mixing proportions 
% given by $q_i$, while the $q_i$ themselves $(i=1,\ldots,N)$ are drawn from
% a symmetrical $\Dir(\alpha,\ldots,\alpha)$ distribution.
% 
% 
% 
% 
% In this model, Gibbs sampling
% proceeds  using the full conditionals
% \begin{eqnarray}
% q_i|\cdots  &\sim&  \Dir(\alpha_1 + \#\{w_i=1\},\ldots,\alpha_J +
% \#\{w_i=J\} ),~~i=1,\ldots,N
% \nonumber   \\ 
% \rule{0pt}{5ex}
% \theta_{j\ell}|\cdots &\sim &\Dir(\lambda_{j\ell 1} + r_{j\ell 1},
% \ldots, \lambda_{j\ell K_\ell} + r_{j\ell K_\ell}), 
%  \nonumber \\
% & & j = 1,\ldots, J;~~\ell = 1,\ldots, L \nonumber \\
% \rule{0pt}{5ex}
% p(w_{it}=j|\cdots) &=& \frac{q_{ij} \thetaof{j}{y_{it}}}
% {\sum_{k=1}^J q_{ij} \thetaof{k}{y_{it}}}~,~
% i=1,\ldots,N;~j=1,\ldots,J; \nonumber  \\
% & & t = 1,\ldots,2L \nonumber
% \end{eqnarray}
% where $\#\{w_i=j\}$ is the number of alleles in the $i\thh$ individual
% currently allocated to gene pool $j$ and $r_{j\ell k}$ denotes the number
% of alleles of type $k$ at locus $\ell$
% currently allocated to gene pool
% $j$. 
% \citeasnoun{Pritchardetal2000} update $\alpha$ 
% by a Metropolis-Hastings method (Appendix~\ref{sec:MH}). The posterior
% distribution of
% $\alpha$ thus estimated provides some insight into the degree to which
% admixture has occurred across individuals.  
% 
% 
% Learning samples would be available if
% there were substantial prior knowledge about the gene pools contributing
% to the admixture and if known, purebred descendants from them
% were separately sampled.   By assuming any
% effects of genetic drift to be negligible, such samples could be treated
% as learning samples in the mixture model.  The full conditional for
% $\theta_{j\ell}$ would then be modified to include the $n_{j\ell k}$ as
% before.  
% 
% \subsection{Block-updating $w_i$ when $J=2$}
% \label{sec:Polya}
% In many situations involving invasions of exotic species, there is
% substantial prior knowledge that the number of major subpopulations or
% ``gene pools" involved is two---the native population and the invading
% population.  Additionally, many hybrid zones are known to be areas of
% hybridization (admixture) between two species or populations.  Here I
% present novel computations, feasible when only two subpopulations or gene
% pools are involved, that eliminate the explicit need for the variable
% $q=(q_1,\ldots,q_N)$ in implementing a Gibbs sampler.   Such a method
% slightly improves mixing of the chain, but is primarily useful as it
% makes possible Gibbs sampling in a simultaneous mixture and admixture
% analysis as will be described in Section~Yippie!.
% 
% The computations themselves may be derived as follows.  Let $J=2$, so
% that each allele in an individual may have originated from gene
% pool~1 or gene pool~2.  Then, each $q_{i1}$ will follow a
% $\mathrm{Beta}(\alpha,\alpha)$ distribution and $q_{i2} = 1 - q_{i1}$.
% Conditional on
% $q_{i1}$, each $w_{it}$ will then be independently a Bernoulli trial with
% $p(w_{it} = 1|q_{i1}) = q_{i1}$.  Marginalizing over $q_{i1}$ (not
% conditioning on the data) it follows that $\#\{w_i=1\}$ follows a
% beta-binomial distribution with parameters $(\alpha,\alpha)$.  Of course,
% each allele in an individual is uniquely labelled so the elements of
% $w_i$ may be interpreted as following a {\em labelled} beta-binomial
% distribution.  Under such a distribution, the elements of $w_i$ are not
% independent, but they are exchangeable \cite{deFinetti1972},
% and hence their marginal distributions are invariant to permutations of
% their order (and thus the arbitrary order we have imposed upon them is
% acceptable).  
% 
% This labelled beta-binomial sampling mechanism is easily visualized
% by a P\'{o}lya-Eggenberger urn scheme
% \cite{Feller1957,Johnsonetal1993}.  Imagine an urn
% initially filled with $b_1$ balls labelled ``1" and $b_2$ balls labelled
% ``2."  Draw a ball randomly and record
% $w_{i1} = 1$ or $2$ according to the ball's label.  Then replace the
% ball to the urn, adding, at the same time, $c$ more balls of the
% same type (1 or 2) as the ball just drawn.  Repeat the process, assigning
% a value to
% $w_{i2}$ and so forth until $w_{i2L}$ has also been assigned a 1 or
% 2\@.  If
% $b_1$,
% $b_2$, and $c$ were chosen to satisfy $b_1/c = b_2/c = \alpha$, then the
% resulting vector $w_i$ would be a realized value from the labelled
% beta-binomial distribution with parameters $(\alpha,\alpha)$.  (One should notice,
% also, that this extends to a non-symmetrical beta distribution,
% say $\mathrm{Beta}(\alpha_1,\alpha_2)$, by choosing $b_1/c = \alpha_1$ and $b_2/c = 
% \alpha_2$.)
% 
% By such a scheme it is apparent that if $d_t$ balls of type~1 have been
% drawn in the first $t$ drawings from the urn, then the
% probability that the next ball drawn is a 1 is given by
% \begin{equation}
% \frac{b_1 + d_t c}{b_1 + b_2 + tc}.
% \label{eq:transprob}
% \end{equation}
% And so the pairs $(w_{it},d_t)$, $t=1,\ldots,2L$, can be interpreted as
% forming a time-inhomogeneous Markov chain in time $t$ with transition
% probabilities determined by (\ref{eq:transprob}) and the obvious
% fact that $d_{t+1} = d_t + 1\{w_{it+1}=1\}$, where $1\{x=a\}$ takes the
% value one when $x=a$ and zero otherwise.  
% 
% The foregoing has all been considered in the absence of data, $y_i$. 
% However, given $\theta$, the data provide some information about the true
% value of each $w_{it}$ by the relation $ p(y_{it}|w_{it},\theta) =
% \thetaof{w_{it}}{y_{it}}$.    Therefore, conditional on $\theta$
% and $y_{it}$, the pairs $(w_{it},d_t)$ participate in a {\em hidden}
% Markov chain. Recognition of this fact allows application of a
% ``filter-forward, simulate-backward" type of algorithm which may be derived following the  
% computations of \citeasnoun{Baumetal1970} in order to
% realize the elements of $w_i$ from their joint full conditional
% distribution, $p(w_i|\alpha, \theta, y_i)$.  Furthermore, using the
% \citeasnoun{Baum1972} algorithm, it is possible to compute
% $p(y_i|\alpha,\theta)$, effectively performing a sum over all possible
% binary vectors of length $2L$ in an efficient manner.  This is described below.
% 
% Take $b_1$, $b_2$, and $c$ as defined above.  Suppressing the $i$
% subscript for clarity, let $w_t \in \{1,2\}$, $t=1,\ldots,2L$, and define
% $d_t = \sum_{\tau=1}^t 1\{w_\tau = 1\}$.  We adopt the notation $w_{\leq
% t}$ ($w_{\geq t}$) to mean $w_1,\ldots,w_{t}$ ($w_t,\ldots,w_{2L}$) for
%  components of $w$, and use the same notation with $y$ and $d$.   The
% pairs
% $(w_t,d_t)$ can be interpreted as following a Markov chain in $t$:
% \begin{eqnarray}
% p(w_{t+1},d_{t+1} | w_{\leq t},d_{\leq t})& = &p(w_{t+1},d_{t+1} | w_{
% t},d_{ t})  \nonumber \\
% &=& 
% \frac{b_1 + d_tc}{b_1+b_2+tc}
% 1\{d_{t+1} = d_t + 1\{w_{t+1} = 1\} \}\nonumber .
% \end{eqnarray}
% The ``perturbed" or ``degraded" observations of the chain are the allelic types 
% $y_1,\ldots,y_{2L}$ which
% depend in hidden Markov fashion on $w$.  For notational clarity, we assume
% implicit dependence on the allele frequencies $\theta$,
% \[
% 	p(y_t | w_{\leq 2L}, d_{\leq 2L}) = p(y_t | w_t) = \thetaof{w_t}{y_t}.
% \]
% This dependence structure is shown in the undirected graph of
% Figure~\ref{fig:undirgraph}.
% \begin{figure}
% \begin{center}
% \includegraphics*[width=.75\textwidth]{undirgraph.eps}
% \caption{An undirected graph showing the dependence between $w$, $d$ and
% $y$ in Section~\ref{sec:Polya}.  This graph describes hidden Markov structure
% for the pairs $(w_t,d_t)$.  The dependence on $\theta$ is implicit
% and not shown. }
% \label{fig:undirgraph}
% \end{center}
% \end{figure}
% 
% \section*{Acknowledgments}
% This work developed out of conversations during a brief visit
% to University of Oxford, where the author was hosted by the lab of Peter Donnelly in the 
% Department of Statistics. Partial support came from
% National Science Foundation Grant BIR--9807747.  The author thanks Jonathan Pritchard, Elizabeth Thompson, and
% Matthew Stephens for helpful discussion and comments on earlier drafts. Special thanks to Mark Beaumont for providing
% the data on Scottish wildcats.
% 
% 
% %% START THE APPENDIX SECTION
% \appendix
% 
% \makeatletter   %% HAVE TO ADD SOMETHING HERE TO MAKE IT SAY "APPENDIX"
%  \renewcommand{\@seccntformat}[1]{APPENDIX~{\csname the#1\endcsname}.\hspace*{1em}}
%  \makeatother
% 
% 
% 
% \section{Metropolis Updates for $\alpha$}
% \label{sec:MH}
% The method of Metropolis sampling is used to update values of $\alpha$.  A new
% value for $\alpha$ denoted $\alpha^*$ is drawn from a proposal distribution.  Since $\alpha$
% is constrained to the interval $(0,A]$, I use a folded normal distribution, centered at
% $\alpha$.  Hence a variable $a$ is drawn from a $\mathrm{Normal}(\alpha,\sigma^2)$
% distribution.  If $0<a\leq A$ then $\alpha^* = a$.  Otherwise if $-A\leq a<0$ then $\alpha^*
% = -a$ and if $A<a\leq 2A$ then $\alpha^* = 2A -a$.  In all other cases ($a<-A$ or $a>2A$) the proposal is
% rejected without further consideration.  The proposal density is then still symmetrical
% \[
% h(\alpha^*|\alpha) = \mathcal{N}(\alpha^*;\alpha,\sigma^2) 
% + \mathcal{N}(-\alpha^*;\alpha,\sigma^2) + \mathcal{N}(2A-\alpha^*;\alpha,\sigma^2)  
% = h(\alpha|\alpha^*)
% \]
% with $\mathcal{N}$ denoting the normal density function.  The standard deviation, $\sigma$, of the
% proposal distribution
%  requires some tuning.  Under model $\Ma$,  $\sigma \approx .12$ seems to
% work well, while when individuals may be purebred or admixed (model $\Mpa$) then $\sigma
% \approx .5$ encourages better mixing with the Scottish cat data.
% 
% The proposed value $\alpha^*$ is accepted as the new value with probability given by the
% minimum of 1 or the  Hastings ratio.   For \citeasnoun{Pritchardetal2000}'s model, using, the
% $q_i$'s, the acceptance probability is 
% \begin{equation}
% \min\left\{1, 
% \frac{\prod_{i=1}^N \mathcal{D}(q_i;\alpha^*,J)}
% {\prod_{i=1}^N \mathcal{D}(q_i;\alpha,J)} \right\}
% \end{equation}
% where $\mathcal{D}(q;\alpha,J)$ denotes the density of a Dirichlet random vector $q$ of $J$
% components with all $J$ parameters equal to $\alpha$.
% 
% When able to eliminate the $q_i$'s (as in Section~\ref{sec:Polya}), then with only admixed
% individuals (model $\Ma$) the acceptance probability may be written as
% \begin{equation}
% \min\left\{1,  
% \frac{\prod_{i=1}^N p(y_i|\alpha^*,\theta)}
% {\prod_{i=1}^N p(y_i|\alpha,\theta)} \right\}.
% \end{equation}
% In the model $M_\mathrm{P,A}$ which includes both purebred and admixed individuals,  the
% acceptance probability is
% \begin{equation}
% \min\left\{1, 
% \frac{\prod_{i=1}^N [\xi_\mathrm{P} p(y_i|\pi,\theta) + \xi_\mathrm{A}
% p(y_i|\alpha^*,\theta)]} {\prod_{i=1}^N [\xi_\mathrm{P} p(y_i|\pi,\theta) + \xi_\mathrm{A}
% p(y_i|\alpha,\theta)]} \right\}.
% \end{equation}
% \mbox{}\vspace*{1ex}
% \mbox{}
% 



%% HERE WE DECLARE THE BIBLIOGRAPHYSTYLE TO USE AND THE BIBLIOGRAPHY DATABASE
\bibliographystyle{ECA_jasa}
\bibliography{example}


\end{document}